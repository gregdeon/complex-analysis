\documentclass{article}

\usepackage{amsmath}
\usepackage{parskip}
\usepackage{tikz}
\renewcommand{\emph}{\textbf}
\renewcommand{\bar}{\overline}
\newcommand{\Log}{\text{Log }}

\begin{document}

\section{Frame 29 -- The Exponential Function}
\subsection{Definition}
We define the \emph{exponential function} $e^z$ by writing
\[
	e^z = e^x e^{iy}
\]
and we apply Euler's formula to get
\[
	e^z = e^x (\cos y + i \sin y)
\]
Note that, when $y = 0$, $e^z$ reduces to $e^x$. 

Although we typically understand that $e^{1/n}$ would be the set of $n$th roots of $e$, here, we only use the real, positive root $\sqrt[n]{e}$.

\subsection{Familar properties}
First, in calculus, we know that
\[
	e^{x_1} e^{x_2} = e^{x_1 + x_2}
\]
It is easy to verify that this holds true for complex numbers:
\[
	e^{z_1} e^{z_2} = e^{z_1 + z_2}
\]
This also allows us to write
\[
	\frac{e^{z_1}}{e^{z_2}} = e^{z_1 - z_2}
\]
and, as a specific case,
\[
	\frac{1}{e^{z}} = e^{-z}
\]

We showed earlier that $e^z$ is differentiable everywhere in the complex plane, and that
\[
	\frac{d}{dz} e^z = e^z
\]
We also know that $e^z$ is never zero. This comes from the pair
\[
	|e^z| = e^x	\quad \text{and} \quad	\text{arg}(e^z) = y + 2n\pi
\]
and since $e^x$ is never zero, neither is $e^z$.

\subsection{Unfamiliar properties}
Since we can write
\[
	e^{z + 2\pi i} = e^z e^{2\pi i} = e^z
\]
the exponential function is periodic with an imaginary period of $2\pi i$.

It is also possible for the complex exponential function to be negative. For an example, we know that Euler's identity states
\[
	e^{i\pi} = -1
\]
In fact, $e^z$ can be any given non-zero complex number.

\textit{Example: suppose we want solutions to the equation
\[
	e^z = 1 + i
\]
The right side can be rewritten as
\[
	e^x e^{iy} = \sqrt{2} e^{i\pi / 4}
\]
and equating the parts of this equation gives
\[
	x = \ln \sqrt{2} = \frac{1}{2} \ln 2 \quad	\text{and} \quad 
	y = \left(2n + \frac{1}{4}\right) \pi
\]
so
\[
	z = \frac{1}{2} \ln 2 + \left( 2n + \frac{1}{4} \right) \pi i
\]}


\clearpage
\section{Frame 30 -- The Logarithmic Function}
\subsection{Motivation}
We said in the previous section that $e^z$ can take on any non-zero complex value. To help us solve the equation
\[
	e^w = z
\]
we will define a \emph{logarithmic function}, such that
\[
	e^{\log z} = z	\quad (z \neq 0)
\]
We can solve for $w$ by writing the two complex numbers in the form
\begin{align*}
	z &= re^{i\theta} \\
	w &= u + iv
\end{align*}
Substituting these into the original equation gives
\[
	e^u e^{iv} = re^{i\theta}
\]
so we get
\[
	w = \log z = \ln r + i(\theta + 2n\pi)
\]
Note that this is a multi-valued function.

\textit{Example: if $z = -1 - i\sqrt{3}$, then $r = 2$ and $\theta = -2\pi/3$, so
\[
	\log(-1 - i\sqrt{3}) = \ln 2 + \left(n - \frac{1}{3}\right) 2\pi i
\]}

\subsection{Precise definition}
A more precise definition of the multi-valued logarithmic function is
\[
	\log z = \ln |z| + i \arg z
\]
The \emph{principal value} of $\log z$ is obtained by using the single-valued principal argument instead:
\[
	\Log z = \ln |z| + i \theta
\]
Note that
\[
	\log z = \Log z + i 2n\pi
\]

\subsection{Notes}
The principal logarithmic function $\Log z$ reduces to the usual logarithm from calculus when $z$ is positive and real -- if $z = r$, then
\[
	\Log r = \ln r
\]
However, we are now able to find the logarithm of negative real numbers, which we were unable to do in calculus.

\textit{Example: the logarithm of $-1$ is
\[
	\log(-1) = \ln 1 + (1 + 2n)i\pi = (2n + 1)i\pi
\]
and
\[
	\Log(-1) = i\pi
\]}

\end{document}