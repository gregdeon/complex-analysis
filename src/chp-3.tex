\documentclass{article}

\usepackage{amsmath}
\usepackage{parskip}
\usepackage{tikz}
\renewcommand{\emph}{\textbf}
\renewcommand{\bar}{\overline}
\DeclareMathOperator{\Log}{Log}
\DeclareMathOperator{\Arg}{Arg}
\DeclareMathOperator{\sech}{sech}
\DeclareMathOperator{\csch}{csch}


\begin{document}

\section{Frame 29 -- The Exponential Function}
\subsection{Definition}
We define the \emph{exponential function} $e^z$ by writing
\[
	e^z = e^x e^{iy}
\]
and we apply Euler's formula to get
\[
	e^z = e^x (\cos y + i \sin y)
\]
Note that, when $y = 0$, $e^z$ reduces to $e^x$. 

Although we typically understand that $e^{1/n}$ would be the set of $n$th roots of $e$, here, we only use the real, positive root $\sqrt[n]{e}$.

\subsection{Familar properties}
First, in calculus, we know that
\[
	e^{x_1} e^{x_2} = e^{x_1 + x_2}
\]
It is easy to verify that this holds true for complex numbers:
\[
	e^{z_1} e^{z_2} = e^{z_1 + z_2}
\]
This also allows us to write
\[
	\frac{e^{z_1}}{e^{z_2}} = e^{z_1 - z_2}
\]
and, as a specific case,
\[
	\frac{1}{e^{z}} = e^{-z}
\]

We showed earlier that $e^z$ is differentiable everywhere in the complex plane, and that
\[
	\frac{d}{dz} e^z = e^z
\]
We also know that $e^z$ is never zero. This comes from the pair
\[
	|e^z| = e^x	\quad \text{and} \quad	\text{arg}(e^z) = y + 2n\pi
\]
and since $e^x$ is never zero, neither is $e^z$.

\subsection{Unfamiliar properties}
Since we can write
\[
	e^{z + 2\pi i} = e^z e^{2\pi i} = e^z
\]
the exponential function is periodic with an imaginary period of $2\pi i$.

It is also possible for the complex exponential function to be negative. For an example, we know that Euler's identity states
\[
	e^{i\pi} = -1
\]
In fact, $e^z$ can be any given non-zero complex number.

\textit{Example: suppose we want solutions to the equation
\[
	e^z = 1 + i
\]
The right side can be rewritten as
\[
	e^x e^{iy} = \sqrt{2} e^{i\pi / 4}
\]
and equating the parts of this equation gives
\[
	x = \ln \sqrt{2} = \frac{1}{2} \ln 2 \quad	\text{and} \quad 
	y = \left(2n + \frac{1}{4}\right) \pi
\]
so
\[
	z = \frac{1}{2} \ln 2 + \left( 2n + \frac{1}{4} \right) \pi i
\]}


\clearpage
\section{Frame 30 -- The Logarithmic Function}
\subsection{Motivation}
We said in the previous section that $e^z$ can take on any non-zero complex value. To help us solve the equation
\[
	e^w = z
\]
we will define a \emph{logarithmic function}, such that
\[
	e^{\log z} = z	\quad (z \neq 0)
\]
We can solve for $w$ by writing the two complex numbers in the form
\begin{align*}
	z &= re^{i\theta} \\
	w &= u + iv
\end{align*}
Substituting these into the original equation gives
\[
	e^u e^{iv} = re^{i\theta}
\]
so we get
\[
	w = \log z = \ln r + i(\theta + 2n\pi)
\]
Note that this is a multi-valued function.

\textit{Example: if $z = -1 - i\sqrt{3}$, then $r = 2$ and $\theta = -2\pi/3$, so
\[
	\log(-1 - i\sqrt{3}) = \ln 2 + \left(n - \frac{1}{3}\right) 2\pi i
\]}

\subsection{Precise definition}
A more precise definition of the multi-valued logarithmic function is
\[
	\log z = \ln |z| + i \arg z
\]
The \emph{principal value} of $\log z$ is obtained by using the single-valued principal argument instead:
\[
	\Log z = \ln |z| + i \theta
\]
Note that
\[
	\log z = \Log z + i 2n\pi
\]

\subsection{Notes}
The principal logarithmic function $\Log z$ reduces to the usual logarithm from calculus when $z$ is positive and real -- if $z = r$, then
\[
	\Log r = \ln r
\]
However, we are now able to find the logarithm of negative real numbers, which we were unable to do in calculus.

\textit{Example: the logarithm of $-1$ is
\[
	\log(-1) = \ln 1 + (1 + 2n)i\pi = (2n + 1)i\pi
\]
and
\[
	\Log(-1) = i\pi
\]}


\clearpage
\section{Frame 31 -- Branches \& Derivatives of Logarithms}
\subsection{Limiting a logarithm's domain}
We saw in the previous section that the multi-valued logarithm function of a complex number $z = re^{i\theta}$ can be written as
\[
	\log z = \ln r + i \theta
\]
where $\theta$ can have any of the values
\[
	\theta = \Arg(z) + 2n\pi
\]
We can make the logarithmic function single-valued by restricting the value of $\theta$ to $\alpha < \theta < \alpha + 2\pi$ for any real value of $\theta$. Then, the function is single-valued and is continuous everywhere in the domain of the function (ie: $r > 0$ and $\theta \in (\alpha, \alpha + 2\pi)$). Note that we cannot include in the ray $\theta = \alpha$ -- the function would not be continuous here.

In this limited domain, the components of the $\log$ function also satisfy the polar Cauchy-Riemann equations
\[
	ru_r = 1 = v_\theta;	\quad	u_\theta = 0 = -rv_r
\]
so the logarithmic function is analytic in this domain, with the derivative
\[
	\frac{d}{dz} \log z = e^{-i\theta}(u_r + iv_r) = \frac{1}{re^{i\theta}}
	= \frac{1}{z}
\]
In particular, we can set $\alpha = -\pi$ and write
\[
	\frac{d}{dz} \Log z = \frac{1}{z}	\quad
	(|z| > 0, - \pi < \Arg z < \pi)
\]

Note that not all of the identities from calculus carry over to the complex plane.

\textit{Example: using the principal branch, 
\[
	\Log(i^3) = \Log(-i) = -i \frac{\pi}{2}
\]
but
\[
	3\Log(i) = 3\left(i\frac{\pi}{2}\right) = i\frac{3\pi}{2}
\]
so
\[
	\Log(i^3) \neq 3\Log(i)
\]}

\subsection{Branches}
A \emph{branch} of a multi-valued function $f$ is any single-valued, analytic function $F$ such that $F(z)$ is one of the values of $f$ at each point within the domain of $F$. For instance, our limited-domain logarithm is a branch of the multi-valued $\log$ function. The principal logarithm function
\[
	\Log z = \ln r + i\theta	\quad
	(|z| > 0, -\pi < \Arg z < \pi)
\]
is known as the \emph{principal branch}.

A \emph{branch cut} is a line/curve that is used to define a branch $F$ of a multi-valued function $f$. Any point on the branch cut is a singular point of $F$. Any point that is common to all branch cuts of $f$ is called a \emph{branch point}. For example, the logarithmic function has a branch point at $z = 0$ and a branch cut on the ray $\theta = \alpha$. In particular, the ray $\theta = \pi$ is the branch cut for the principal logarithmic function.



\clearpage
\section{Frame 32 -- Logarithm Identities}
We said earlier that arguments, which are multi-valued functions, can be compared in a special way -- since each function is really a set of values, the sets will contain the same values. Specifically,
\[
	\arg(z_1 z_2) = \arg(z_1) + \arg(z_2)
\]
Now, we know that $|z_1z_2| = |z_1||z_2|$, and from our knowledge of real-valued logarithms,
\[
	\ln|z_1 z_2| = \ln|z_1| + \ln|z_2|
\]
Putting together these two statements, we see that
\[
	\log(z_1 z_2) = \log z_1 + \log z_2
\]
which is to be understood as \textit{set equality}, and does not necessarily apply to the principal values. In a similar manner,
\[
	\log \left( \frac{z_1}{z_2} \right) = \log z_1 - \log z_2
\]

Two more properties will be useful in the next section. If $z$ is any non-zero complex number, then
\[
	z^n = e^{n \log z}
\]
for all values of $\log z$. When $n = 1$, this reduces to the familiar
\[
	z = e^{\log z}
\]

Also, for any non-zero $z$, it is true that
\[
	z^{1 / n} = e^{\frac{1}{n} \log z}
\]
where both sides have $n$ distinct values. To show this, we can write out the right side as
\[
	e^{\frac{1}{n} \log z}
	= e^{\frac{1}{n} \ln r + \frac{i (\theta + 2k\pi}{n}}
	= \sqrt[n]{r} e^{i(\theta/n + 2k\pi/n)}
\]
which has $n$ distinct values, for $k = 0, 1, \dots, n - 1$.


\clearpage
\section{Frame 33 -- Complex Exponents}
\subsection{Definition and basics}
For non-zero $z$ and complex $c$, we define the function $z^c$ as
\[
	z^c = e^{c \log z}
\]
Note that this definition uses the multi-valued $\log$ function.

We saw earlier that the exponential function has the property
\[
	\frac{1}{e^z} = e^{-z}
\]
Now, for the general power equation, we have
\[
	\frac{1}{z^c} = \frac{1}{e^{c \log z}} = e^{-c \log z} = z^{-c}
\]

\textit{Example: the values of $i^{-2i}$ can be found by first writing
\[
	\log i = \ln 1 + i\left( \frac{\pi}{2} + 2n\pi\right)
	= i \left( 2n + \frac{1}{2} \right) \pi
\]
and so
\[
	i^{-2i} = e^{-2i \cdot i(2n + 1/2)\pi}
	= e^{(4n + 1) \pi}
\]
Note that all of these powers are real numbers.}

The \emph{principal value} of $z^c$ uses the single-valued log function:
\[
	\text{P.V. } z^c = e^{c \Log z}
\]

\textit{Example: the principal value of $(-i)^i$ is
\[
	e^{i \Log(-i)}
	= e^{i(-i \pi/2)}
	= e^{\pi / 2}
\]}

\subsection{Other properties}
To differentiate $z^c$, we can restrict the logarithmic function to a single branch
\[
	\log z = \ln + i\theta	\quad (r > 0, \alpha < \theta < \alpha + 2\pi)
\]
Then, $z^c$ is analytic in this domain. The derivative can be found through the chain rule:
\[
	\frac{d}{dz} z^c
	= \frac{d}{dz} e^{c \log z}
	= \frac{c}{z} e^{c \log z}
	= c z^{c-1}
\]

Most of the laws of exponents remain valid in the complex plane. However, since the functions are multi-valued, we can only guarantee equality between sets -- when using principal values, not all of the rules of real exponents work. For example, the law
\[
	z_1^c z_2^c = (z_1 z_2)^c
\]
does not necessarily hold for all $z_1, z_2$ when using principal values.

\subsection{Exponential functions with other bases}
We can write the \emph{exponential function} with a non-zero base $c$ as
\[
	c^z = e^{z \log c}
\]
Note that this function is, again, multi-valued: if $c = e$, then we don't recover our usual definition of $e^z$. However, if we use the principal value of the logarithm, the usual interpretation occurs.

This exponential function is an entire function for any non-zero $c$. It has the derivative
\[
	\frac{d}{dz} c^z 
	= \frac{d}{dz} e^{z \log c}
	= e^{z \log c} \log c
	= c^z \log c
\]


\clearpage
\section{Frame 34 -- Trigonometric Functions}
\subsection{Trigonometric functions -- definitions}
We know from Euler's formula that
\begin{align*}
	e^{ix}  &= \cos x + i \sin x \\
	e^{-ix} &= \cos x - i \sin x
\end{align*}
and we can rearrange this into
\begin{align*}
	\sin x &= \frac{e^{ix} - e^{-ix}}{2i} \\
	\cos x &= \frac{e^{ix} + e^{-ix}}{2 }
\end{align*}

In a similar manner, we can define the complex \emph{trigonometric functions}
\begin{align*}
	\sin z &= \frac{e^{iz} - e^{-iz}}{2i} \\
	\cos z &= \frac{e^{iz} + e^{ iz}}{2}
\end{align*}

\subsection{Trigonometric functions -- properties}
First, the two trig functions are entire, since they are linear combinations of two entire functions. Thus, they are differentiable everywhere; their derivatives, from the complex exponential derivatives, are
\begin{align*}
	\frac{d}{dz} \sin z &= \cos z \\
	\frac{d}{dz} \cos z &= -\sin z
\end{align*}
We can also see that the odd/even properties carry over:
\begin{align*}
	\sin(-z) &= - \sin z \\
	\cos(-z) &=   \cos z
\end{align*}
and Euler's formula also applies:
\[
	e^{iz} = \cos z + i \sin z
\]

Many of the identities from trigonometry carry over. A sample of these identities is:
\begin{align*}
	\sin(z_1 + z_2) &= \sin z_1 \cos z_2 + \cos z_1 \sin z_2 \\
	\cos(z_1 + z_2) &= \cos z_1 \cos z_2 - \sin z_1 \sin z_2 \\
	\sin 2z &= 2 \sin z \cos z \\
	\cos 2z &= \cos^2 z - \sin^2 z \\
	\sin \left( z + \frac{\pi}{2} \right) &=   \cos z \\
	\sin \left( z - \frac{\pi}{2} \right) &= - \cos z \\
	\sin (z +   \pi) &= -\sin z \\
	\sin (z + 2 \pi) &=  \sin z \\
	\cos (z +   \pi) &= -\cos z \\
	\cos (z + 2 \pi) &=  \cos z \\
	\sin^2 z + \cos^2 z &= 1
\end{align*}

\subsection{Using hyperbolic functions}
The hyperbolic trig functions of a real number $y$ are defined, from calculus, as
\begin{align*}
	\sinh y &= \frac{e^y - e^{-y}}{2} \\
	\cosh y &= \frac{e^y + e^{-y}}{2}
\end{align*}
We can use these definitions to write
\begin{align*}
	\sin(iy) &= i \sinh y \\
	\cos(iy) &=   \cosh y
\end{align*}
Then, if $z = x + iy$ is a complex number, we can write
\begin{align*}
	\sin z &= \sin x \cosh y + i \cos x \sinh y \\
	\cos z &= \cos x \cosh y - i \sin x \sinh y
\end{align*}
These expressions allow us to write
\begin{align*}
	|\sin z|^2 &= \sin^2 x + \sinh^2 y \\
	|\cos z|^2 &= \cos^2 x + \sinh^2 y
\end{align*}
and, since $\sinh y$ is unbounded in $y$, the trigonometric functions are \emph{unbounded} on the complex plane.

\subsection{Extensions to other trigonometric functions}
Since
\[
	|\sin z|^2 = \sin^2 x + \sinh^2 y
\]
we find that $\sin z$ only has zeroes at $x = n \pi$ and $y = 0$; ie:
\[
	\sin z = 0 \iff z = n\pi
\]
Since $\cos z = -\sin(z - \pi/2$, we find that 
\[
	\cos z = 0 \iff z = \left( \frac{1}{2} + n \right) \pi
\]

With these zeroes in mind, we can define the four other trigonometric functions as expected:
\begin{align*}
	\tan z &= \frac{\sin z}{\cos z} \\
	\sec z &= \frac{1}{\cos z} \\
	\cot z &= \frac{\cos z}{\sin z} \\
	\csc z &= \frac{1}{\sin z}
\end{align*}
These functions are analytic everywhere except for the singularities caused by the denominators: $\tan$ and $\sec$ are analytic for all $z \neq (n + 1/2)\pi$, and $\cot$ and $\csc$ are analytic for all $z \neq n\pi$. 

We can use our differentiation rules to find the expected differentiation formulas:
\begin{align*}
	\frac{d}{dz} \tan z &= \sec^2 z \\
	\frac{d}{dz} \sec z &= \sec z \tan z \\
	\frac{d}{dz} \cot z &= -\csc^2 z \\
	\frac{d}{dz} \csc z &= -\csc z \cot z
\end{align*}


\clearpage
\section{Frame 35 -- Hyperbolic Trigonometry}
\subsection{Definitions}
Following suit from the previous section, we can define the \emph{hyperbolic sine and cosine} of a complex variable as
\begin{align*}
	\sinh z &= \frac{e^z - e^{-z}}{2} \\
	\cosh z &= \frac{e^z + e^{-z}}{2}
\end{align*}
As with the regular trigonometric functions, these are linear combinations of entire functions, so they are also entire, with the derivatives
\begin{align*}
	\frac{d}{dz} \sinh z &= \cosh z \\
	\frac{d}{dz} \cosh z &= \sinh z
\end{align*}

\subsection{Properties}
Due to the similar definitions of the regular and hyperbolic trig functions, we can write the following relationships:
\begin{align*}
	\cosh(iz) &= \cos z \\
	\cos (iz) &= \cosh z \\
	-i \sinh(iz) &= \sin z \\
	-i \sin (iz) &= \sinh z
\end{align*}
Some common identities are:
\begin{align*}
	\sinh(-z) &= - \sinh z \\
	\cosh(-z) &=   \cosh z \\
	\cosh^2 z - \sinh^2 z &= 1 \\ 
	\sinh(z_1 + z_2) &= \sinh z_1 \cosh z_2 + \cosh z_1 \sinh z_2 \\
	\cosh(z_1 + z_2) &= \cosh z_1 \cosh z_2 + \sinh z_1 \sinh z_2 \\
	\sinh z &= \sinh x \cos y + i \cosh x \sin y \\
	\cosh z &= \cosh x \cos y + i \sinh x \sin y \\
	| \sinh z |^2 &= \sinh^2 x + \sin^2 y \\
	| \cosh z |^2 &= \sinh^2 x + \cos^2 y
\end{align*}
We can obtain most of these identities by converting the hyperbolic trig functions into regular trig (as above) and applying the identities discussed in the previous section.

\subsection{Extensions}
Since $\sin$ and $\cos$ are periodic with a period of $2\pi$, it is clear that $\sinh$ and $\cosh$ are also periodic with a period of $2\pi i$. Extending this, we can find the zeroes of each function:
\begin{align*}
	\sinh z = 0 &\iff z = n\pi i \\
	\cosh z = 0 &\iff z = (n + 1/2) \pi i
\end{align*}

We can also define the remaining four hyperbolic trig functions as
\begin{align*}
	\tanh z &= \frac{\sinh z}{\cosh z} \\
	\sech z &= \frac{1}{\cosh z} \\
	\coth z &= \frac{\cosh z}{\sinh z} \\
	\csch z &= \frac{1}{\sinh z} \\
\end{align*}
Again, using the quotient rule, we find that their derivatives are
\begin{align*}
	\frac{d}{dz} \tanh z &= \sech^2 z \\
	\frac{d}{dz} \sech z &= - \sech z \tanh z \\
	\frac{d}{dz} \coth z &= - \csch^2 z \\
	\frac{d}{dz} \csch z &= - \csch z \coth z
\end{align*}

\end{document}