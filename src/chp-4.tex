\documentclass{article}

\usepackage{amsmath}
\usepackage{parskip}
\usepackage{tikz}
\renewcommand{\emph}{\textbf}
\renewcommand{\bar}{\overline}
\DeclareMathOperator{\Log}{Log}
\DeclareMathOperator{\Arg}{Arg}
\DeclareMathOperator{\sech}{sech}
\DeclareMathOperator{\csch}{csch}

\begin{document}

\section{Frame 37 -- Derivatives with Real Variables}
\subsection{Definition}
In the previous chapter, we looked at derivatives of complex functions of a complex variable $z$. Now, we look at the derivatives of a complex-valued function of a real variable $t$. If we write our function as
\[
	w(t) = u(t) + iv(t)
\]
where $u$ and $v$ are real-valued, then we can define the derivative of $w$ at a point $t$ as
\[
	w'(t) = \frac{d}{dt} w(t) = u'(t) + iv'(t)
\]
provided that $u'$ and $v'$ exist at $t$.

\subsection{Properties}
If $z_0 = x_0 + iy_0$ is a complex constant, then we can show that
\begin{align*}
	\frac{d}{dt} [z_0 w(t)]
	&= [(x_0 + iy_0)(u(t) + iv(t)]' \\
	&= [x_0 u(t) - y_0 v(t)]' + i[y_0 u(t) + x_0 v(t)]' \\
	&= [x_0 u'(t) - y_0 v'(t)] + i[y_0 u'(t) + x_0 v'(t)] \\
	&= z_0 w'(t)
\end{align*}
as we expect.

Next, if $z_0$ is still a complex constant, the derivative of $e^{z_0 t}$ is
\begin{align*}
	\frac{d}{dt} e^{z_0 t} 
	&= \frac{d}{dt} e^{x_0 t}(\cos y_0 t + i sin y_0 t) \\
	&= \frac{d}{dt} e^{x_0 t} \cos y_0 t + i \frac{d}{dt} e^{x_0 t} \sin y_0 t \\
	&= (x_0 + iy_0)(e^{x_0 t} \cos y_0 t + ie^{x_0 t} \sin y_0 t) \\
	&= z_0 e^{z_0 t}
\end{align*}

Many other rules carry over from standard calculus. However, some rules no longer apply. For instance, in calculus, the mean value theorem for derivatives states that
\[
	w'(c) = \frac{w(b) - w(a)}{b - a}
\]
for some $c$ in the interval $a \le c \le b$ as long as $w$ is continuous. However, this is easily disproved by the function
\[
	w(t) = e^{it}
\]
If $a = 0$ and $b = 2\pi$, then $w(a) = w(b) = 1$ and we expect to find a point $c$ in $[0, 2\pi]$ such that $w'(c) = 0$. However, no such points exist -- the magnitude of the derivative is always $1$.


\clearpage
\section{Frame 38 -- Definite Integrals of Complex Functions}
\subsection{Definitions}
If $w(t)$ is a complex-valued function of a real variable $t$, as in the previous section
\[
	w(t) = u(t) + iv(t)
\]
then we define the \emph{definite integral} of $w(t)$ over the interval $a \le t \le b$ as
\[
	\int_a^b w(t) dt = \int_a^b u(t) dt + i \int_a^b v(t) dt
\]
provided the two right-side integrals exist. Then,
\begin{align*}
	\Re\left[\int_a^b w(t)dt\right] &= \int_a^b \Re[w(t)] dt \\
	\Im\left[\int_a^b w(t)dt\right] &= \int_a^b \Im[w(t)] dt
\end{align*}
Improper integrals over unbounded intervals are defined similarly.

The two real integrals will exist as long as $u$ and $v$ are \emph{piecewise continuous} on the interval $[a, b]$ -- that is, continuous everywhere in the interval except possibly for a finite number of points where it has one-sided limits. When $u$ and $v$ are piecewise continuous, we say that $w$ is also piecewise continuous.

\subsection{Properties}
The most common rules of integrals from calculus apply here as well:
\begin{itemize}
	\item $\int z_0 w(t)dt = z_0 \int w(t)$
	
	\item $\int w_1(t) + w_2(t)dt = \int w_1(t)dt + \int w_2(t)dt$
	
	\item $\int_a^b w(t) dt = -\int_b^a w(t) dt$
	
	\item $\int_a^b w(t) dt = \int_a^c w(t) dt + \int_c^b w(t) dt$
\end{itemize}

We can also extend the fundamental theorem of calculus to complex integrals. Suppose that two functions
\begin{align*}
	w(t) &= u(t) + iv(t) \\
	W(t) &= U(t) + iV(t) 
\end{align*}
are continuous on the interval $[a, b]$ and $W'(t) = w(t)$ when $a \le t \le b$. Then, we can write
\[
	\int_a^b w(t) dt = W(b) - W(a) = W(t) \Big|_a^b 
\] 

\textit{Example: noting that the derivative of $\frac{1}{i} e^{it}$ is
\[
	\frac{d}{dt} \left( \frac{1}{i} e^{it} \right)
	= \frac{1}{i} ie^{it}
	= e^{it}
\]
we can evaluate $\int e^{it} dt$ as
\begin{align*}
	\int_0^{\pi/4} e^{it} dt 
	&= \frac{e^{it}}{i} \Big|_0^{\pi/4} \\
	&= \frac{1}{i} \left[ e^{\pi/4} - 1 \right] \\
	&= \frac{1}{i} \left[ \frac{1}{\sqrt{2}} - 1 + \frac{i}{\sqrt{2}} \right] \\
	&= \frac{1}{\sqrt{2}} + i\left( 1 - \frac{1}{\sqrt{2}} \right)
\end{align*}}

As in the previous section, the mean value theorem for integrals does not apply. We can show this by finding the integral $\int_0^{2\pi} e^{it} dt = 0$, even though the function is never zero on this interval.




\end{document}