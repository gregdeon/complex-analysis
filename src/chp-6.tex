\documentclass{article}

\usepackage{amsmath}
\usepackage{parskip}
\usepackage{tikz}
\renewcommand{\emph}{\textbf}
\renewcommand{\bar}{\overline}
\DeclareMathOperator{\Log}{Log}
\DeclareMathOperator{\Arg}{Arg}
\DeclareMathOperator{\sech}{sech}
\DeclareMathOperator{\csch}{csch}
\DeclareMathOperator{\Res}{Res}

\begin{document}

\section{Frame 68 -- Isolated Singular Points}
\subsection{Definition}
Earlier, we defined a \emph{singular point} of a function $f$ as a point $z_0$ where $f$ is not analytic, but $f$ is analytic at some point in every neighbourhood of $z_0$. Additionally, we will define an \emph{isolated} singular point as such a point where there exists a deleted neighbourhood on which $f$ is analytic.

\subsection{Examples}
\textit{The function
\[
	f(z) = \frac{z+1}{z^3 (z^2 + 1)}
\]
has three singular points at $z = 0$ and $z = \pm i$. These three points are isolated singularities.}

\textit{The principal branch of the logarithm
\[
	\Log z = \ln r + i\Theta 	\quad (-\pi < \theta < \pi)
\]
has a singular point at $z = 0$. However, this is not an isolated singular point, since every neighbourhood of $0$ also contains some points on the negative real axis, where the function is not analytic.}

\textit{The function
\[
	f(z) = \frac{1}{\sin(\pi / z)}
\]
has singular points at $z = 0$ and $z = 1/n$ for all integers $n$. All of these singular points are isolated except for the one at $z = 0$ -- every neighbourhood of $0$ also contains a point $z = 1/m$ because we can find such a point
\[
	0 < 1/m < \epsilon
\]
for each $\epsilon$.}

\subsection{Important Points}
Note that if a function has a finite number of singular points, then all of these must be isolated -- we can make a neighbourhood around each singular point that does not contain any others.

Finally, note that we may also refer to the point at infinity as an isolated singular point. This happens if there exists a positive $R_1$ such that there are no singularities in the region
\[
	R_1 < |z| < \infty
\]


\clearpage
\section{Frame 69 -- Residues}
\subsection{Definition}
Suppose that $f$ is a function with an isolated singular point at $z_0$. Then, there exists some deleted neighbourhood
\[
	0 < |z - z_0| < R_2
\]
where $f$ is analytic. On this domain, we can express $f(z)$ as the Laurent series
\[
	f(z) = \sum_{n=0}^\infty a_n (z - z_0)^n
	+ \frac{b_1}{z - z_0}
	+ \frac{b_2}{(z - z_0)^2}
	+ \dots
	+ \frac{b_n}{(z - z_0)^n}
	+ \dots
\]
These coefficients come from various integral representations. In particular,
\[
	b_n = \frac{1}{2\pi i} \int_C \frac{f(z)}{(z - z_0)^{-n+1}} dz
\]
where $C$ is a positively oriented, simple, closed contour around $z_0$ in the deleted neighbourhood described above. When $n = 1$, this expression becomes
\[
	\int_C f(z) dz = 2\pi i b_1
\]

We say that the complex number $b_1$ is called the \emph{residue} of $f$ at the isolated singular point $z_0$, and we write
\[
	b_1 = \Res_{z = z_0} f(z)
\]
so the contour integral around $z_0$ becomes
\[
	\int_C f(z) dz = 2\pi i \Res_{z = z_0} f(z)
\]

\subsection{Examples}
\textit{Example 1: We can evaluate the integral
\[
	\int_C z^2 \sin\left( \frac{1}{z} \right) dz
\]
where $C$ is the positively-oriented unit circle. First, we note that the integrand is analytic everywhere except the origin, so the Laurent series converges on $0 < |z| < \infty$. Then, we can write
\begin{align*}
	z^2 \sin(1 / z)
	&= z^2 \sum_{n=0}^\infty \frac{(-1)^n}{(2n+1)!} (1 / z)^{2n+1} \\
	&= \sum_{n=0}^\infty \frac{(-1)^n}{(2n+1)!} z^{1 - 2n}
\end{align*}
so the coefficient $b_1$ is $-1/3!$. Thus,
\[
	\int_C z^2 \sin\left( \frac{1}{z} \right) dz
	= 2\pi i \left( -\frac{1}{3!} \right) 
	= -\frac{\pi i}{3}
\]}

\textit{Example 2: We can repeat the previous problem for the integral
\[
	\int_C e^{1 / z^2}
\]
Since this series is
\begin{align*}
	e^{1 / z^2}
	&= \sum_{n=0}^\infty \frac{(1 / z^2)^n}{n!} \\
	&= \sum_{n=0}^\infty \frac{1}{n! z^{2n}} \\
	&= 1 + \frac{1}{z^2} + \frac{1}{2 z^4} + \dots 
\end{align*}
the residue at $0$ is $0$, so
\[
	\int_C e^{1 / z^2} = 0
\]}

\textit{Example 3: Finally, we can use residues to evaluate the integral
\[
	\int_C \frac{1}{z(z - 2)^4} dz
\]
around the positively-oriented circle $|z - 2| = 1$. The Laurent series is
\begin{align*}
	\frac{1}{z(z - 2)^4}
	&= \frac{1}{(z - 2)^4} \frac{1}{2 + (z - 2)} \\
	&= \frac{1}{2(z - 2)^4} \frac{1}{1 - \frac{-(z - 2)}{2}} \\
	&= \frac{1}{2(z - 2)^4} \sum_{n=0}^\infty \frac{(-1)^n}{2^n} (z - 2)^n \\
	&= \sum_{n=0}^\infty \frac{(-1)^n}{2^{n+1}} (z - 2)^{n-4}
\end{align*}
Thus, the coefficient of $1 / (z - 2)$ is at $n = 3$, and
\[
	b_1 = -\frac{1}{16}
\]
so
\[
	\int_C \frac{1}{z(z - 2)^4} dz = 2\pi i\left(-\frac{1}{16}\right) 
	= -\frac{\pi i}{8}
\]}


\end{document}