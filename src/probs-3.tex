\documentclass{article}

\usepackage{amsmath}
\usepackage{parskip}
\usepackage{tikz}
\renewcommand{\emph}{\textbf}
\renewcommand{\bar}{\overline}
\DeclareMathOperator{\Log}{Log}
\DeclareMathOperator{\Arg}{Arg}


\begin{document}

\section{Frame 29 -- The Exponential Function}
\textbf{1(a)}
The value is
\[
	e^{2 \pm 3\pi i} 
	= e^2 e^{\pm 3\pi i}
	= e^2 (-1)
	= e^2
\]

\textbf{1(b)}
The value is
\[
	e^{1/2 + i \pi/4}
	= e^{1/2} e^{i \pi 4}
	= \sqrt{e} \frac{1 + i}{\sqrt{2}}
	= \sqrt{\frac{e}{2}} (1 + i)
\]

\textbf{1(c)}
This expression can be split up as
\[
	e^{z + \pi i}
	= e^z e^{i \pi}
	= e^z \cdot -1
	= -e^z
\]

\textbf{3}
The function can be written as
\[
	e^{\bar{z}}
	= e^{x} e^{-iy}
	= e^x \cos y - i e^x \sin y
\]
so the partial derivatives are
\begin{align*}
	u_x &=  e^x \cos y \\
	u_y &= -e^x \sin y \\
	v_x &= -e^x \sin y \\
	v_y &= -e^x \cos y
\end{align*}
These don't satisfy Cauchy-Riemann, so the function is differentiable nowhere and thus analytic nowhere.

\textbf{8(a)}
The equation $e^z = -2$ can be written as
\[
	e^z = 2e^{i\pi}
\]
so
\begin{align*}
	e^x = 2				&\to x = \ln 2 \\
	e^{iy} = e^{i\pi}	&\to y = (2n + 1) \pi
\end{align*}
and
\[
	z = \ln 2 + (2n + 1) i\pi
\]

\textbf{8(b)}
The equation $e^z = 1 + i\sqrt{3}$ can be written as
\[
	e^z = 2 e^{i \pi/3}
\]
so
\begin{align*}
	x &= \ln 2 \\
	y &= \left( \frac{1}{3} + 2n \right) \pi
\end{align*}
and
\[
	z = \ln 2 + \left( \frac{1}{3} + 2n \right) i \pi
\]

\textbf{8(c)}
If
\[
	e^{2z - 1} = 1
\]
then
\[
	2z - 1 = i2n\pi
\]
or, solving for $z$,
\[
	z = \frac{1}{2} + i n\pi
\]



\clearpage
\section{Frame 31 -- Logarithms}
\textbf{1(a)}
Evaluating the logarithm,
\[
	\Log(-ei) = \ln(e) + \Arg(-ei) = 1 - i\frac{\pi}{2}
\]

\textbf{1(b)}
As above,
\[
	\Log(1 - i) = \ln(\sqrt{2}) + \Arg(1 - i)
	= \frac{1}{2}\ln(2) - i\frac{\pi}{4}
\]

\textbf{2(a)}
The set of values is
\[
	\log(e) = \ln(e) + \arg(e) 
	= 1 + i2n\pi
\]

\textbf{2(b)}
As above,
\[
	\log(i) = \ln(1) + \arg(i)
	= 0 + i\left(2n + \frac{1}{2}\right) \pi
\]

\textbf{2(c)}
As above,
\[
	\log(-1 + i\sqrt{3}) = \ln(\sqrt{4}) + \arg(-1 + i\sqrt{3})
	= \ln 2 + i\left(2n + \frac{2}{3}\right) \pi
\]

\textbf{3(a)}
The left side is
\[
	\Log(1 + i)^2 = \ln(2) + i \frac{\pi}{2}
\]
and the right side is
\[
	2\Log(1 + i) = 2 \left[ \frac{1}{2} \ln(2) + i\frac{\pi}{4} \right]
	= \ln(2) + i\frac{\pi}{2}
	= \Log(1 + i)^2
\]

\textbf{3(b)}
The left side is
\[
	\Log(-1 + i)^2 = \ln(2) - i \frac{\pi}{2}
\]
and the right side is
\[
	2 \Log(-1 + i) = 2 \left[ \frac{1}{2} \ln(2) + i\frac{3\pi}{4} \right]
	= \ln(2) + i\frac{3\pi}{2}
\]

\textbf{5(a)}
Since $i^{1/2}$ can be written as the set
\[
	i^{1/2} = e^{\pi/4 + \pi k}		\quad (k = 0, \pm 1, \pm 2, \dots)
\]
it has the logarithm
\[
	\log(i^{1/2}) = \ln(1) + i \arg(i^{1/2})
	= i \left( n + \frac{1}{4} \right) \pi
\]
Then,
\[
	\frac{1}{2} \log i
	= \frac{1}{2} \left[ \ln(1) + i\left(2n + \frac{1}{2}\right)\pi \right]
	= i\left( n + \frac{1}{4} \right) \pi
	= \log(i^{1/2})m
\]

\textbf{6}
Differentiating and using the chain rule, the identity
\[
	z = e^{\log z}
\]
becomes
\[
	1 = e^{\log z} \cdot \frac{d}{dz} \log z
\]
or
\[
	\frac{d}{dz} \log z = \frac{1}{e^{\log z}} = \frac{1}{z}
\]

\textbf{7}
To solve the equation
\[
	\log z = i\frac{\pi}{2}
\]
we write
\[
	z = e^{i \pi/2} = i
\]

\end{document}