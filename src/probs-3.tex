\documentclass{article}

\usepackage{amsmath}
\usepackage{parskip}
\usepackage{tikz}
\renewcommand{\emph}{\textbf}
\renewcommand{\bar}{\overline}
\DeclareMathOperator{\Log}{Log}
\DeclareMathOperator{\Arg}{Arg}


\begin{document}

\section{Frame 29 -- The Exponential Function}
\textbf{1(a)}
The value is
\[
	e^{2 \pm 3\pi i} 
	= e^2 e^{\pm 3\pi i}
	= e^2 (-1)
	= e^2
\]

\textbf{1(b)}
The value is
\[
	e^{1/2 + i \pi/4}
	= e^{1/2} e^{i \pi 4}
	= \sqrt{e} \frac{1 + i}{\sqrt{2}}
	= \sqrt{\frac{e}{2}} (1 + i)
\]

\textbf{1(c)}
This expression can be split up as
\[
	e^{z + \pi i}
	= e^z e^{i \pi}
	= e^z \cdot -1
	= -e^z
\]

\textbf{3}
The function can be written as
\[
	e^{\bar{z}}
	= e^{x} e^{-iy}
	= e^x \cos y - i e^x \sin y
\]
so the partial derivatives are
\begin{align*}
	u_x &=  e^x \cos y \\
	u_y &= -e^x \sin y \\
	v_x &= -e^x \sin y \\
	v_y &= -e^x \cos y
\end{align*}
These don't satisfy Cauchy-Riemann, so the function is differentiable nowhere and thus analytic nowhere.

\textbf{8(a)}
The equation $e^z = -2$ can be written as
\[
	e^z = 2e^{i\pi}
\]
so
\begin{align*}
	e^x = 2				&\to x = \ln 2 \\
	e^{iy} = e^{i\pi}	&\to y = (2n + 1) \pi
\end{align*}
and
\[
	z = \ln 2 + (2n + 1) i\pi
\]

\textbf{8(b)}
The equation $e^z = 1 + i\sqrt{3}$ can be written as
\[
	e^z = 2 e^{i \pi/3}
\]
so
\begin{align*}
	x &= \ln 2 \\
	y &= \left( \frac{1}{3} + 2n \right) \pi
\end{align*}
and
\[
	z = \ln 2 + \left( \frac{1}{3} + 2n \right) i \pi
\]

\textbf{8(c)}
If
\[
	e^{2z - 1} = 1
\]
then
\[
	2z - 1 = i2n\pi
\]
or, solving for $z$,
\[
	z = \frac{1}{2} + i n\pi
\]



\clearpage
\section{Frame 31 -- Logarithms}
\textbf{1(a)}
Evaluating the logarithm,
\[
	\Log(-ei) = \ln(e) + \Arg(-ei) = 1 - i\frac{\pi}{2}
\]

\textbf{1(b)}
As above,
\[
	\Log(1 - i) = \ln(\sqrt{2}) + \Arg(1 - i)
	= \frac{1}{2}\ln(2) - i\frac{\pi}{4}
\]

\textbf{2(a)}
The set of values is
\[
	\log(e) = \ln(e) + \arg(e) 
	= 1 + i2n\pi
\]

\textbf{2(b)}
As above,
\[
	\log(i) = \ln(1) + \arg(i)
	= 0 + i\left(2n + \frac{1}{2}\right) \pi
\]

\textbf{2(c)}
As above,
\[
	\log(-1 + i\sqrt{3}) = \ln(\sqrt{4}) + \arg(-1 + i\sqrt{3})
	= \ln 2 + i\left(2n + \frac{2}{3}\right) \pi
\]

\textbf{3(a)}
The left side is
\[
	\Log(1 + i)^2 = \ln(2) + i \frac{\pi}{2}
\]
and the right side is
\[
	2\Log(1 + i) = 2 \left[ \frac{1}{2} \ln(2) + i\frac{\pi}{4} \right]
	= \ln(2) + i\frac{\pi}{2}
	= \Log(1 + i)^2
\]

\textbf{3(b)}
The left side is
\[
	\Log(-1 + i)^2 = \ln(2) - i \frac{\pi}{2}
\]
and the right side is
\[
	2 \Log(-1 + i) = 2 \left[ \frac{1}{2} \ln(2) + i\frac{3\pi}{4} \right]
	= \ln(2) + i\frac{3\pi}{2}
\]

\textbf{5(a)}
Since $i^{1/2}$ can be written as the set
\[
	i^{1/2} = e^{\pi/4 + \pi k}		\quad (k = 0, \pm 1, \pm 2, \dots)
\]
it has the logarithm
\[
	\log(i^{1/2}) = \ln(1) + i \arg(i^{1/2})
	= i \left( n + \frac{1}{4} \right) \pi
\]
Then,
\[
	\frac{1}{2} \log i
	= \frac{1}{2} \left[ \ln(1) + i\left(2n + \frac{1}{2}\right)\pi \right]
	= i\left( n + \frac{1}{4} \right) \pi
	= \log(i^{1/2})m
\]

\textbf{6}
Differentiating and using the chain rule, the identity
\[
	z = e^{\log z}
\]
becomes
\[
	1 = e^{\log z} \cdot \frac{d}{dz} \log z
\]
or
\[
	\frac{d}{dz} \log z = \frac{1}{e^{\log z}} = \frac{1}{z}
\]

\textbf{7}
To solve the equation
\[
	\log z = i\frac{\pi}{2}
\]
we write
\[
	z = e^{i \pi/2} = i
\]


\clearpage
\section{Frame 33 -- Complex Exponents}
\textbf{1(a)}
The values of $(1 + i)^i$ are
\[
	(1 + i)^i
	= e^{i \cdot (\ln \sqrt{2} + i(2n + 1/4)\pi)}
	= e^{i \ln (2) / 2} e^{-\pi/4 + 2n\pi}
\]

\textbf{1(b)}
The values of $(-1)^{1/\pi}$ are
\[
	(-1)^{1/pi}
	= e^{\frac{1}{\pi} \cdot i(2n + 1)\pi}
	= e^{(2n + 1)i}
\]

\textbf{2(a)}
The principal value of $i^i$ is
\[
	i^i
	= e^{i \Log i}
	= e^{i (i \pi/2)}
	= e^{-\pi / 2}
\]

\textbf{2(b)}
The principal value of $[\frac{e}{2}(-1 - i\sqrt{3})]^{3\pi i}$ is
\[
	[\frac{e}{2}(-1 - i\sqrt{3})]^{3\pi i}
	= \exp\left[3\pi i \left( 1 - i\frac{2\pi}{3}\right) \right]
	= e^{3\pi i} e^{-2\pi^2}
	= -e^{-2\pi^2}
\]

\textbf{2(c)}
The principal value of $(1 - i)^{4i}$ is
\[
	(1 - i)^{4i}
	= e^{4i (\frac{1}{2} \ln 2 - i\pi/4)}
	= e^{i 2\ln(2)} e^{\pi}
	= e^{\pi} [\cos(2\ln2) + i\sin(2\ln2)]
\]

\textbf{3}
The values of $(-1 + i\sqrt{3})^{3/2}$ are
\[
	(-1 + i\sqrt{3})^{3/2}
	= e^{3/2 \cdot (\ln 2 + i \pi(2n + 2/3))}
	= 2^{3/2} e^{i (\pi + 3n\pi)}
	= \pm 2\sqrt{2}
\]


\clearpage
\section{Frame 34 -- Trigonometric Functions}
\textbf{15}
To find all roots of the equation
\[
	\sin z = \cosh 4
\]
we can write
\[
	\sin z = \sin x \cosh y + i \cos x \sinh y
\]
Then, equating real and imaginary parts, we find that
\begin{align*}
	\sin x \cosh y &= \cosh 4 \\
	\cos x \sinh y &= 0
\end{align*}
The second equation says that $x = (n + 1/2) \pi$. Since $\sin x$ here is one of $\pm 1$, the first equation becomes
\[
	\cosh y = \pm \cosh 4
\]
However, $\cosh y > 0$ for all $y$ snd $\cosh y = \cosh (-y)$, so the only solutions are
\[
	z = \left(2n + \frac{1}{2} \right) \pi \pm 4i
\]

\textbf{16}
To find all roots of the equation
\[
	\cos z = 2
\]
we can write
\[
	\cos z = \cos x \cosh y - i \sin x \sinh y
\]
so
\begin{align*}
	\cos x \cosh y &= 2	\\
	\sin x \sinh y &= 0
\end{align*}
The second equation states that $x = n\pi$ or $y = 0$. The second case has no solutions, so we must have $x = n\pi$. Then, $\cos x = \pm 1$. As above, we require that $\cos x = 1$ for any solutions to be found. This gives
\[
	x = 2n\pi 
\]
so
\[
	y = \cosh^{-1}(2)
\]
or, putting these together,
\[
	z = 2n\pi + i \cosh^{-1}(2)
\]

To simplify this, we can try to find a simpler expression for $y$. If $\cosh y = 2$, then
\[
	\frac{e^{y} + e^{-y}}{2} = 2
\]
or
\[
	(e^y)^2 - 4(e^y) + 1 = 0
\]
This has solutions when
\[
	e^y = 2 \pm \sqrt{3}
\]
or
\[
	y = \ln(2 \pm \sqrt{3})
\]
However, 
\[
	\frac{1}{2 - \sqrt{3}} = 2 + \sqrt{3}
\]
so
\[
	\ln(2 + \sqrt{3}) = -\ln(2 - \sqrt{3})
\]
and the solutions are
\[
	z = 2n\pi \pm i \ln(2 + \sqrt{3})
\]


\clearpage
\section{Frame 35 -- Hyperbolic Trigonometry}
\textbf{1}
Since $\sinh z$ is defined as
\[
	\sinh z = \frac{e^z - e^{-z}}{2}
\]
its derivative is
\[
	\frac{d}{dz} \sinh z 
	= \frac{d}{dz} \frac{e^z - e^{-z}}{2}
	= \frac{e^z + e^{-z}}{2}
	= \cosh z
\]

\textbf{2}
Using the definitions,
\[
	\sinh 2z = \frac{e^{2z} - e^{-2z}}{2}
	= 2\frac{e^z + e^{-z}}{2} \frac{e^z - e^{-z}}{2}
	= 2 \cosh z \sinh z
\]

\textbf{4}
Since $z = x + iy$, we can write
\[
	\sinh z = \sinh(x + iy)
	= \sinh(x) \cosh(iy) + \sinh(iy) \cosh(x)
	= \sinh(x) \cos(y) + i \cosh(x) \sin(y)
\]

\textbf{8}
The zeroes of $\sinh$ are at 
\begin{align*}
	\sinh x \cos y &= 0 \\
	\cosh x \sin y &= 0
\end{align*}
The second equation only holds when $y = n\pi$. At all of these points, $\cos y \neq 0$, so the first equation only holds when $x = 0$. The zeroes, then, occur at
\[
	z = 0 + in\pi
\]

Since $\cosh z = \sinh(z + \pi / 2)$, the zeroes of $\cosh$ are at
\[
	z = 0 + i(n + 1/2) \pi
\] 

\textbf{15(a)}
Splitting $\sinh$ into its real and imaginary components, the equation is
\begin{align*}
	\sinh x \cos y &= 0 \\
	\cosh x \sin y &= 1
\end{align*}
The first equation says that $x = 0$ or $y = (n + 1/2)\pi$. In the first case, $\cosh(0) = 1$, so the second equation simply becomes
\[
	\sin y = 1
\]
which has solutions at $y = (2n + 1/2) \pi$. In the second case, $\sin(y) = \pm 1$; examining these solutions, they result in the same set of numbers. Finally, putting the parts together, the solutions are at
\[
	z = 0 + i(2n + 1/2)\pi
\]

\textbf{15(b)}
Splitting $\cosh$ into its real and imaginary parts, the equation is
\begin{align*}
	\cosh x \cos y &= \frac{1}{2} \\
	\sinh x \sin y &= 0
\end{align*}
The second equation has two solutions:
\begin{itemize}
	\item $x = 0$: Here, $\cosh(0) = 1$, so the top equation becomes
	\[
		\cos y = \frac{1}{2}
	\]
	which has solutions at
	\[
		y = \begin{cases}
			\frac{ \pi}{3} + 2n\pi, \\
			\frac{-\pi}{3} + 2n\pi
		\end{cases}
	\]
	
	\item $y = n\pi$: Here, $\cos y = \pm 1$, so
	\[
		\cosh x = \pm \frac{1}{2}
	\]
	which has no solutions.
\end{itemize}
Overall, the solutions are the set
\[
	z = 0 + i\left( 2n \pm \frac{1}{3}\right)\pi
\]

\textbf{16}
Equating the real and imaginary parts,
\begin{align*}
	\cosh x \cos y &= -2 \\
	\sinh x \sin y &= 0
\end{align*}
As in the previous problem, there are two solutions to the second equation:
\begin{itemize}
	\item $x = 0$: The first equation reduces to
	\[
		\cos y = -2
	\]
	which has no solutions.
	
	\item $y = n\pi$: Depending on $y$, this reduces the first equation to
	\[
		\cosh x = \pm 2
	\]
	This only has solutions in the positive case, where
	\[
		x = \cosh^{-1} 2
	\]
\end{itemize}
The solution set is simply
\[
	z = \cosh^{-1} 2 + i(2n + 1)\pi
\]
Thinking back to the previous section, this can be simplified to
\[
	z = \pm \ln(2 + \sqrt{3}) + i(2n + 1)\pi
\]


\clearpage
\section{Frame 36 -- Inverse Trigonometry}
\textbf{1(a)}
The values of $\tan^{-1}(2i)$ are
\begin{align*}
	\arctan(2i) 
	&= \frac{i}{2} \log \frac{3i}{-i} \\
	&= \frac{i}{2} \log {-3} \\
	&= \frac{i}{2} [\ln 3 + i(2n + 1)\pi] \\
	&= (n + 1/2)\pi + i \frac{\ln 3}{2}
\end{align*}

\textbf{1(b)}
The values of $\tan^{-1}(1 + i)$ are
\begin{align*}
	\arctan(1 + i)
	&= \frac{i}{2} \log \frac{1 + 2i}{-1} \\
	&= \frac{i}{2} \log -(1 + 2i) \\
	&= \frac{i}{2} [\ln \sqrt{5} + i(2n + \arctan(2) + 1)\pi] \\
	&= [n - (1 + \arctan 2 )]\pi + \frac{i}{4} \ln 5  
\end{align*}

\textbf{1(c)}
Since $(-1)^2 - 1 = 0$, the values of $\cosh^{-1}(-1)$ are
\[
	\cosh^{-1}(-1) = \log(-1)
	= i(2n + 1)\pi
\]

\textbf{1(d)}
The values of $\tanh^{-1}(0)$ are
\begin{align*}
	\tanh^{-1}(0)
	&= \frac{1}{2} \log 1 \\
	&= \frac{1}{2} (2n\pi i) \\
	&= i n\pi
\end{align*}

\textbf{2}
The solutions to the equation $\sin z = 2$ are
\begin{align*}
	z &= \sin^{-1} 2 \\
	&= -i \log [2i + (-3)^{1/2}] \\
	&= -i \log [i (2 \pm \sqrt{3})]
\end{align*}
The values of this logarithm are
\[
	\log[i (2 + \sqrt{3})]
	= \ln(2 + \sqrt{3}) + i(2n + 1/2)\pi 
\]
and
\[
	\log[i (2 - \sqrt{3})]
	= \ln(2 - \sqrt{3}) + i(2n + 1/2)\pi
	= -\ln(2 + \sqrt{3}) + i(2n + 1/2)\pi
\]
or, combining these two sets,
\[
	\log[i (2 \pm \sqrt{3})]
	= \pm \ln(2 + \sqrt{3}) + i(2n + 1/2)\pi
\]
Then,
\[
	\sin^{-1} 2
	= \left(2n + \frac{1}{2} \right) \pi \pm i\ln(2 + \sqrt{3})
\]

\textbf{3}
The values of $\cos^{-1} \sqrt{2}$ are
\[
	\cos^{-1} \sqrt{2}
	= -i \log[\sqrt{2} + i(-1)^{1/2}]
	= -i \log[\sqrt{2} \pm 1]
\]
This logarithm has two values:
\[
	\log[\sqrt{2} + 1] 
	= \ln(\sqrt{2} + 1) + i2n\pi
\]
and
\[
	\log[\sqrt{2} - 1]
	= \ln(\sqrt{2} - 1) + i2n\pi
	= -\ln(\sqrt{2} + 1) + i2n\pi
\]
so the solutions are
\[
	\cos^{-1} \sqrt{2}
	= 2n\pi \pm i \ln(1 + \sqrt{2})
\]

\end{document}