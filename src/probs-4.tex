\documentclass{article}

\usepackage{amsmath}
\usepackage{parskip}
\usepackage{tikz}
\renewcommand{\emph}{\textbf}
\renewcommand{\bar}{\overline}
\DeclareMathOperator{\Log}{Log}
\DeclareMathOperator{\Arg}{Arg}
\DeclareMathOperator{\sech}{sech}
\DeclareMathOperator{\csch}{csch}

\begin{document}

\section{Frame 38 -- Derivatives and Integrals}
\textbf{1(b)}
Breaking the derivative into its complex components,
\begin{align*}
	\frac{d}{dt} [w(t)]^2
	&= \frac{d}{dt} [u(t) + iv(t)]^2 \\
	&= 2[u(t) + iv(t)][u(t) + iv(t)]' \\
	&= 2w(t)w'(t)
\end{align*}

\textbf{2(a)}
Evaluating the integral,
\begin{align*}
	\int_1^2 \left( \frac{1}{t} - i \right)^2 dt 
	&= \int_1^2 \frac{1}{t^2} - 1 - i \frac{2}{t} dt \\
	&= - \frac{1}{t} - t - 2i \ln t \Big|_1^2 \\
	&= - (\frac{1}{2} - 1) - (2 - 1) - 2i (\ln 2 - 0) \\
	&= - \frac{1}{2} - i\ln 4
\end{align*}

\textbf{2(b)}
\begin{align*}
	\int_0^{\pi/6} e^{i2t} 
	&= \frac{1}{2i} e^{i2t} \Big|_0^{\pi/6} \\
	&= \frac{1}{2i} (e^{i\pi/3} - 1) \\
	&= \frac{1}{2i} \left(i \frac{\sqrt{3}}{2} - \frac{1}{2} \right) \\
	&= \frac{\sqrt{3}}{4} + i \frac{1}{4}
\end{align*}

\textbf{2(c)}
Converting this improper integral into a limit,
\begin{align*}
	\int_0^\infty e^{-zt} 
	&= \lim_{L \to \infty} \int_0^L e^{-zt} \\
	&= \lim_{L \to \infty} - \frac{1}{z} e^{-zt} \Big|_0^L \\
	&= - \frac{1}{z} \lim_{L \to \infty} e^{-zL} - 1 \\
	&= \frac{1}{z}
\end{align*}

\textbf{4}
Evaluating the left-side integral,
\begin{align*}
	\int_0^\pi e^{1+i}x dx 
	&= \frac{1}{1+i} e^{1+i}x \Big|_0^\pi \\
	&= \frac{1}{1+i} (e^{\pi + i\pi} - 1) \\
	&= \frac{1 - i}{2} (-e^\pi - 1) \\
	&= -\frac{1}{2} (e^\pi + 1) + \frac{i}{2} (e^\pi + 1)
\end{align*}


\clearpage
\section{Frame 39 -- Contours}
\textbf{2}
First, the original parametrization can be written as
\[
	z(\theta) = 2e^{i\theta} = 2\cos \theta + 2i\sin \theta
\]
If $\theta = \arctan \frac{y}{\sqrt{4 - y^2}}$, then this becomes
\[
	z(y) = 2\cos \arctan \frac{y}{\sqrt{4 - y^2}} 
	+ 2i\sin \arctan \frac{y}{\sqrt{4 - y^2}}
\]
Next, these terms can be simplified using basic geometry. The expression $\arctan \frac{y}{\sqrt{4 - y^2}}$ represents a right-angled triangle with legs of lengths $\sqrt{4 - y^2}$ and $y$, so the hypotenuse must have a length of $2$. Then,
\begin{align*}
	\cos \arctan \frac{y}{\sqrt{4 - y^2}} &= \frac{\sqrt{4 - y^2}}{2} \\
	\sin \arctan \frac{y}{\sqrt{4 - y^2}} &= \frac{y}{2} 
\end{align*}
so the arc is
\[
	z(y) = \sqrt{4 - y^2} + i y
\]

\textbf{6}
\textbf{(a)}
First, the function
\[
	z(t) = t + iy(t)
	= t + it^3 \sin(\pi /t)
\]
intersects the real axis whenever $y(t) = 0$. If $t = 1/n$, then this expression becomes
\[
	y(1/n) = \frac{\sin\left(\frac{\pi}{1 / n} \right)}{n^3}
	= \frac{\sin(n\pi)}{n^3}
	= 0
\]
as predicted.

\textbf{(b)}
An arc is smooth if the function $z(t)$ is continuous and its derivative is piecewise continuous.

First, $z(t)$ is continuous for $0 < x \le 1$ because $x(t) = x$ and $y(t) = y(x)$ are both continuous on this interval. To show continuity at $t = 0$, we must show that
\[
	\lim_{t \to 0+} y(t) = 0
\]
However, the magnitude of $y(t)$ must be in the range
\[
	0 \le \left| t^3 \sin\left( \frac{\pi}{t} \right)\right| \le t^3
\]
and the left- and right-hand limits are
\begin{align*}
	\lim_{t \to 0+} 0 &= 0 \\
	\lim_{t \to 0+} t^3 &= 0
\end{align*}
so, by the squeeze theorem, the original limit holds, and $y(t)$ is continuous at $t = 0$.

Finally, the derivative of $z(t)$ is
\[
	z'(t) = 1 + i\left[3t^2 \sin(\pi / t) - \pi t \cos(\pi / t) \right]	
\]
Using the same process as above, the limit as $t$ goes to zero is
\[
	\lim_{t \to 0+} z'(t) = 1 + i0
\]
\textit{How can I tell whether this is continuous? The derivative isn't defined at zero.}


\clearpage
\section{Frame 42 -- Contour Integrals}
\textbf{1(a)}
The integrand on this circle is
\[
	f[z(\theta)] = \frac{2e^{i\theta} + 2}{2e^{i\theta}}
	= 1 + e^{-i\theta}
\]
and the derivative of the contour is
\[
	z'(\theta) = 2ie^{i\theta}
\]
so the first contour integral is
\begin{align*}
	\int_C f(z) dz 
	&= \int_0^\pi (1 + e^{-i\theta})(2ie^{i\theta}) d\theta \\
	&= 2i \int_0^\pi e^{i\theta} + 1 d\theta \\
	&= 2i \left( \frac{e^{i\theta}}{i} + \theta \right)_0^\pi \\
	&= 2i \left( \frac{-2}{i} + \pi \right) \\
	&= -4 + 2\pi i
\end{align*}

\textbf{1(b)}
The integrand has not changed, so
\begin{align*}
	\int_C f(z) dz
	&= 2i \left( \frac{e^{i\theta}}{i} + \theta \right)_\pi^{2\pi} \\
	&= 4 + 2\pi i
\end{align*}

\textbf{1(c)}
The integral along the entire circle is just the sum of the two previous results:
\begin{align*}
	\int_C f(z) dz
	&= (-4 + 2\pi i) + (4 + 2\pi i) \\
	&= 4\pi i
\end{align*}

\textbf{2(a)}
The function is
\[
	f[z(\theta)] = 1 + e^{i\theta} - 1 = e^{i\theta}
\]
and the contour's derivative is
\[
	z'(\theta) = ie^{i\theta}
\]
so the integral is
\begin{align*}
	\int_C f(z) dz
	&= \int_\pi^{2\pi} e^{i\theta} \cdot ie^{i\theta} \\
	&= i \int_\pi^{2\pi} e^{2i\theta} \\
	&= i \frac{e^{2i\theta}}{2i} \Big|_{\pi}^{2\pi} \\
	&= \frac{e^{4i\pi} - e^{2i\pi}}{2} \\
	&= 0
\end{align*}

\textbf{2(b)}
Now, the function is
\[
	f[z(x)] = x - 1
\]
and the path's derivative is
\[
	z'(x) = 1
\]
so the integral is
\[
	\int_C f(z) dz 
	= \int_0^2 x - 1 dx
	= \frac{(x - 1)^2}{2} \Big|_0^2
	= \frac{1 - 1}{2} 
	= 0
\]

\textbf{3}
Along the first edge, the integral is
\[
	\int_0^1 \pi e^{\pi x} dx 
	= e^{\pi x} \Big|_0^1
	= e^\pi - 1
\]

Along the second edge, the integral is
\[
	e^\pi \int_0^1 i \pi e^{-i \pi y} dy
	= e^\pi \left( -\frac{\pi}{\pi} e^{-i \pi y}\right)_0^1
	= 2e^\pi 
\]

On the third edge, the integral is
\[
	- \int_0^1 \pi e^{\pi x} \cdot (-1) dx 
	= e^\pi - 1
\]

Finally, on the fourth edge, the integral is
\[
	-\int_0^1 i \cdot \pi e^{-i \pi y} dy
	= \left( e^{-i\pi y} \right)_0^1
	= -2
\]
so the whole integral is
\[
	\int_C f(z) dz = 2(e^\pi - 1) + 2e^\pi - 2
	= 4(e^\pi - 1)
\]

\textbf{4}
If the path is $y = x^3$, then the direction is
\[
	z'(x) = 1 + iy'(x) = 1 + i3x^2
\]
Then, the integral can be done in two parts. First, from $x = -1$ to $0$,
\[
	\int_{-1}^0 1 \cdot (1 + i3x^2) dx 
	= \left(x + ix^3 \right)_{-1}^0
	= 1 + i
\]
Then, from $x = 0$ to $1$,
\[
	\int_0^1 4x^3 \cdot (1 + i3x^2) dx
	= \int_0^1 4x^3 + i12x^5 dx
	= x^4 + i2x^6 \Big|_0^1
	= 1 + 2i
\]
so the total contour integral is
\[
	\int_C f(z) dz = 2 + 3i
\]

\textbf{5}
If $f(z) = 1$, then
\[
	\int_C f(z) dz
	= \int_a^b 1 \cdot z'(t) dt
	= z(b) - z(a) 
	= z_2 - z_1
\]
	
\textbf{6}
If $z = e^{i\theta}$, then the function $f(z)$ is
\[
	f[z(\theta)] 
	= (e^{i\theta})^{-1 + i}
	= e^{-i\theta} e^{-\theta}
\]
and the contour integral is
\begin{align*}
	\int_C f(z) dz
	&= \int_0^{2\pi} e^{-i\theta} e^{-\theta} \cdot ie^{i\theta} d\theta \\
	&= i \int_0^{2\pi} e^{-\theta} d\theta \\
	&= -i e^{-\theta} \Big|_0^{2\pi} \\
	&= i (1 - e^{-2\pi})
\end{align*}


\textbf{7}
On this semicircle,
\[
	f[z(\theta)] = (e^{i\theta})^i = e^{-\theta}
\]
so
\begin{align*}
	\int_C f(z) dz
	&= \int_0^\pi e^{-\theta} \cdot ie^{i\theta} d\theta \\
	&= i \int_0^\pi e^{(-1 + i) \theta} d\theta \\
	&= \frac{i}{-1 + i} e^{(-1 + i) \theta} \Big|_0^\pi \\
	&= \frac{i(-1 - i)}{2} ( -e^{-\pi} - 1 ) \\
	&= \frac{-1 + i}{2}(e^{-\pi} + 1)
\end{align*}


\clearpage
\section{Frame 43 -- Bounding Contour Integrals}
\textbf{1}
On the contour,
\[
	|z^2 - 1| \le ||z^2| - 1| = 3
\]
so
\[
	|f(z)| \le \frac{1}{3}
\]
and the integral must satisfy
\[
	\left| \int_C \frac{dz}{z^2 - 1} \right| \le \frac{1}{3} \cdot \pi = \frac{\pi}{3} 
\]


\textbf{2}
On the line segment from $z = i$ to $z = 1$, the function $z^4$ is minimized at the midpoint:
\[
	|z^4| 
	\ge \left| \frac{1 + i}{2} \right|^4 
	= \left(\frac{1}{\sqrt{2}} \right)^4
	= \frac{1}{4}
\]
so
\[
	|f(z)| 
	\le \frac{1}{|z^4|}
	\le 4
\]
Then, since the line segment has a length of $L = \sqrt{2}$,
\[
	\left| \int_C \frac{dz}{z^4} \right| \le 4 \cdot \sqrt{2}
\]


\textbf{3}
On these three line segments, $|e^z|$ is maximized when $\Re z$ is maximized, so
\[
	| e^z | \le e^0 = 1
\]
Next, $|\bar{z}|$ is maximized when $|z|$ is maximized, so
\[
	| \bar{z} | \le | -4 | = 4
\]
Thus,
\[
	|f(z)|
	= |e^z - \bar{z}|
	\le |e^z| + |\bar{z}|
	\le 1 + 4
	= 5
\]
Finally, the three line segments have a total length of $12$, so
\[
	\left| \int_C (e^z - \bar{z}) dz \right| \le 5 \cdot 12 = 60
\]
\end{document}