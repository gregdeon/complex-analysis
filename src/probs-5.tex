\documentclass{article}

\usepackage{amsmath}
\usepackage{parskip}
\usepackage{tikz}
\renewcommand{\emph}{\textbf}
\renewcommand{\bar}{\overline}
\DeclareMathOperator{\Log}{Log}
\DeclareMathOperator{\Arg}{Arg}
\DeclareMathOperator{\sech}{sech}
\DeclareMathOperator{\csch}{csch}

\begin{document}

\section{Frame 56 -- Sequences and Series}

\textbf{3}
If
\[
	\lim_{n \to \infty} z_n = z
\]
then, for some integer $n_0$, all of the terms $z_k$ ($k > n_0$) will be in some $\epsilon$ neighbourhood of $z$; ie:
\[
	|z_k - z| < \epsilon
\]
However,
\[
	||z_k| - |z| \le \epsilon
\]
so all of the terms $|z_k|$ must be inside the same $\epsilon$ neighbourhood of $|z|$, and we can say that
\[
	\lim_{n \to \infty} |z_n| = |z|
\]

\textbf{4}
Starting from the series
\[
	\sum_{n=1}^\infty z^n = \frac{1}{1 - z} - 1 = \frac{z}{1 - z}
\]
the components of this expression can be written as
\begin{align*}
	\frac{z}{1 - z}
	&= \frac{r \cos \theta + i r \sin \theta}{1 - r \cos \theta - i r \sin \theta} \\
	&= \frac{r\cos \theta - r^2 \cos^2 \theta - r^2 \sin^2 \theta}{(1 - r\cos \theta)^2 + r^2 \sin^2 \theta}
	+ i \frac{r\sin \theta - r^2 \sin \theta \cos \theta + r^2 \sin \theta \cos \theta}{(1 - r\cos \theta)^2 + r^2 \sin^2 \theta} \\
	&= \frac{r \cos \theta - r^2}{1 - 2r\cos \theta + r^2}
	+ i\frac{r\sin \theta}{1 - 2r\cos \theta + r^2}
\end{align*}
so, equating the real and imaginary parts of the sum,
\[
	\sum_{n=1}^\infty r^n \cos n\theta
	= \frac{r \cos \theta - r^2}{1 - 2r\cos \theta + r^2}
\]
and
\[
	\sum_{n=1}^\infty r^n \sin n\theta
	= \frac{r\sin \theta}{1 - 2r\cos \theta + r^2}
\]


\clearpage
\section{Frame 59 -- Taylor Series}
\textbf{1}
The Maclaurin series for $z \cosh(z^2)$ is
\[
	z \cosh(z^2)
	= z \cdot \sum_{n=0}^\infty \frac{(z^2)^{2n}}{(2n)!}
	= \sum_{n=0}^\infty \frac{z^{4n+1}}{(2n)!}
\]

\textbf{3}
The Maclaurin series for this function is
\[
	\frac{z}{9} \frac{1}{1 + (z^4/9)}
	= \frac{z}{9} \sum_{n=0}^\infty (-1)^n (z^4/9)^n
	= \sum_{n=0}^\infty \frac{(-1)^n z^{4n+1}}{9^{n+1}}
\]

\textbf{4}
Starting with the Maclaurin series for $\sin z$, this function's expansion is
\[
	\sin(z^2)
	= \sum_{n=0}^\infty \frac{(-1)^n}{(2n+1)!} (z^2)^{2n + 1}
	= \sum_{n=0}^\infty \frac{(-1)^n}{(2n+1)!} z^{4n+2}
\]
Thus, $a_k$ is only non-zero for $k = 2, 6, 10, 14, \dots$.

\textbf{10}
The function $\tanh z$ has singularities wherever $\cosh z = 0$, which occurs at $z = (k + 1/2) \pi i$. Thus, the closest singularity has a radius of $\pi/2$, and this is the radius of convergence.

We can find some of the terms of the Taylor series. The constant is
\[
	\tanh(0) = 0
\]
The first derivative is
\[
	\frac{d}{dz} \tanh(z) \Big|_{z=0}
	= \sech^2(0)
	= 1
\]
The second derivative is
\[
	\frac{d}{dz} \sech^2(z) \Big|_{z=0}
	= 2\sech(z) \cdot (-\sech z \tanh z) \Big|_{z=0}
	= -2\sech^2 z \tanh z \Big|_{z=0}
	= 0
\]
The third derivative is
\[
	\frac{d}{dz} -2\sech^2 z \tanh z \Big|_{z=0}
	= -2 \left[ -2\sech^2 z \tanh^2 z + \sech^4 z \right] \Big|_{z=0}
	= -2
\]
so the first few terms of the Taylor series are
\[
	\tanh z \approx z - \frac{z^3}{3}
\]

\textbf{11(a)}
The series for this function is
\[
	\frac{e^z}{z^2}
	= \frac{1}{z^2} \sum_{n=0}^\infty \frac{z^n}{n!}
	= \sum_{n=0}^\infty \frac{z^{n-2}}{n!}
	= \frac{1}{z^2} + \frac{1}{z} + \sum_{n=0}^\infty \frac{z^n}{(n+2)!}
\]

\textbf{11(b)}
The series for this function is
\begin{align*}
	\frac{\sin(z^2)}{z^4}
	&= \frac{1}{z^4} \sum_{n=0}^\infty \frac{(-1)^n}{(2n+1)!} (z^2)^{2n+1} \\
	&= \sum_{n=0}^\infty \frac{(-1)^n}{(2n+1)!} z^{4n - 2} \\
	&= \frac{1}{z^2} + \sum_{n=0}^\infty \frac{(-1)^{n+1}}{(2n+3)!} z^{4n+2}
\end{align*}

\end{document}