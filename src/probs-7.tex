\documentclass{article}

\usepackage{amsmath}
\usepackage{parskip}
\usepackage{tikz}
\renewcommand{\emph}{\textbf}
\renewcommand{\bar}{\overline}
\DeclareMathOperator{\Log}{Log}
\DeclareMathOperator{\Arg}{Arg}
\DeclareMathOperator{\sech}{sech}
\DeclareMathOperator{\csch}{csch}
\DeclareMathOperator{\Res}{Res}

\begin{document}
\section{Frame 79 -- Evaluating Improper Integrals}
\textbf{1}
First, along a semi-circular contour $|z| = R$,
\[
	\int_{C_R} \frac{dx}{x^2 + 1}
	\le \pi R \cdot \frac{1}{R^2 - 1}
	= \pi \frac{R}{R^2 - 1}
\]
and this vanishes as $R$ approaches infinity.

Then, we can find the residue at $z = i$ as
\[
	\Res_{z=i} \frac{1}{z^2 + 1}
	= \frac{1}{z + i} \Big|_{z=i}
	= -\frac{i}{2}
\]
and so
\[
	\int_0^\infty \frac{dz}{z^2 + 1}
	= \frac{1}{2} \left[2\pi i \cdot \frac{-i}{2} \right]
	= \frac{\pi}{2}
\]

\textbf{2}
The integral along the upper semicircle is
\[
	\int_{C_R} \frac{dx}{(x^2 + 1)^2} 
	\le \pi R \cdot \frac{1}{(R^2 - 1)^2}
\]
which vanishes for large $R$. Then,
\begin{align*}
	\int_0^\infty \frac{dx}{(x^2 + 1)^2}
	&= \frac{1}{2} 2\pi i \Res_{z=i} \frac{1}{(z^2 + 1)^2} \\
	&= \pi i \left[ \frac{d}{dz} \frac{1}{(z + i)^2} \Big|_{z=i} \right] \\
	&= \pi i \left[ \frac{-2}{(2i)^3} \right] \\
	&= \frac{\pi}{4}
\end{align*}

\textbf{3}
The integral along the upper semicircle vanishes as above. Then, there are singular points at $c_k = e^{i\pi/4}, e^{i3\pi/4}$. The residue at these points is
\[
	\Res_{z=c_k} \frac{1}{z^4 + 1}
	= \frac{1}{4c_k^3}
\]
or, specifically,
\begin{align*}
	\Res_{z=e^{i \pi/4}} f(z) &= \frac{1}{4} e^{-i3\pi/4} \\
	&= \frac{1}{4\sqrt{2}} (-1 - i) \\ 
	\Res_{z=e^{i3\pi/4}} f(z) &= \frac{1}{4} e^{-i \pi/4} \\
	&= \frac{1}{4\sqrt{2}} ( 1 - i) 
\end{align*}
so
\[
	\sum B_k = \frac{1}{4\sqrt{2}} (-2i) = \frac{-i}{2\sqrt{2}}
\]
Finally,
\[
	\int_0^\infty \frac{dx}{x^4 + 1} 
	= \frac{1}{2} 2\pi i \sum B_k
	= \pi i \cdot \frac{-i}{2\sqrt{2}}
	= \frac{\pi}{2\sqrt{2}}
\]

\textbf{4}
The limit
\[
	\lim_{R \to \infty} \frac{R^3}{(R^2 + 1)}{(R^2 + 4)}
\]
is zero, so the integral along the semicircular contour vanishes. Then, this function has simple poles at $z = i, 2i$; the residue at each is
\begin{align*}
	\Res_{z=i} \frac{z^2}{(z^2 + 1)(z^2 + 4)}
	&= \frac{z^2}{(z+i)(z^2 + 4)} \Big|_{z=i} \\
	&= \frac{-1}{(2i)(3)} \\
	&= \frac{i}{6} \\
	\Res_{z=2i} \frac{z^2}{(z^2 + 1)(z^2 + 4)}
	&= \frac{z^2}{(z^2+1)(z + 2i)} \Big|_{z=2i} \\
	&= \frac{-4}{(-3)(4i)} \\
	&= -\frac{i}{3} 
\end{align*}
so the sum of the residues is $\sum B_k = -i/6$. Then, the improper integral is
\[
	\int_0^\infty \frac{x^2}{(x^2 + 1)(x^2 + 4)}
	= \frac{1}{2} 2\pi i \frac{-i}{6}
	= \frac{\pi}{6}
\]

\textbf{5}
The limit
\[
	\lim_{R \to \infty} \frac{R^3}{(R^2 - 9)(R^2 - 4)^2} = 0
\]
allows us to neglect the semicircular contour. Then, the function has singular points at $z = 2i, 3i$. The first of these residues is
\begin{align*}
	B_1
	&= \Res_{z = 2i} \frac{z^2}{(z^2 + 9)(z^2 + 4)^2} \\
	&= \frac{d}{dz} \frac{z^2}{(z^2 + 9)(z + 2i)^2} \Big|_{z = 2i} \\
	&= \frac{d}{dz} \frac{z^2}{(z^2 + 9)(z^2 + 4iz - 4)}  \Big|_{z = 2i} \\
	&= \frac{d}{dz} \frac{z^2}{z^4 + 4iz^3 + 5z^2 + 36iz - 36} \Big|_{z = 2i} \\
	&= \frac{2z(z^2 + 9)(z + 2i)^2 - z^2(4z^3 + 12iz^2 + 10z + 36i)}
		{(z^2 + 9)^2(z + 2i)^4} \Big|_{z = 2i} \\
	&= \frac{4i(5)(-16) - (-4)(-32i - 48i + 20i + 36i)}{(25)(256)} \\
	&= \frac{-320i - 96i}{6400} \\
	&= \frac{-13i}{200}
\end{align*}
and the second is
\begin{align*}
	B_2
	&= \Res_{z = 3i} \frac{z^2}{(z^2 + 9)(z^2 + 4)^2} \\
	&= \frac{z^2}{(z + 3i)(z^2 + 4)^2} \Big|_{z = 3i} \\
	&= \frac{-9}{(6i)(-5)^2} \\
	&= \frac{12i}{200}
\end{align*}
so this integral is
\[
	\int_0^\infty \frac{x^2}{(x^2 + 9)(x^2 + 4)^2}
	= \pi i \cdot \frac{-i}{200}
	= \frac{\pi}{200}
\]

\textbf{6}
The function
\[
	f(z) = \frac{1}{z^2 + 2z + 2}
\]
has simple poles at $z = -1 \pm i$. The residue at the top point is
\[
	B
	= \Res_{z = -1+i} \frac{1}{z^2 + 2z + 2}
	= \frac{1}{2(-1 + i) + 2} 
	= \frac{1}{2i}
\]
so the principal value of the integral is
\[
	\text{P.V.} \int_{-\infty}^\infty \frac{dx}{x^2 + 2x + 2}
	= 2\pi i \cdot \frac{1}{2i}
	= \pi
\]

\clearpage
\section{Frame 81 -- Fourier Integrals}
\textbf{1}
The function
\[
	f(z) = \frac{1}{(x^2 + a^2)(x^2 + b^2)}
\]
has singular poles at $z = \pm ia, \pm ib$. The residue of the function $f(z) e^{iz}$ at the upper two points is
\begin{align*}
	\Res_{z=ia} f(z) e^{iz}
	&= \frac{e^{-a}}{2ia(b^2 - a^2)} \\
	&= \frac{-e^{-a}}{2ia(a^2 - b^2)} \\
	\Res_{z=ib} f(z) e^{iz}
	&= \frac{e^{-b}}{(a^2 - b^2)2ib} \\
\end{align*}
and the integral along the semicircular contour vanishes, leaving
\[
	\int_{-\infty}^\infty f(x) \cos x~dx 
	= \Re\left(2\pi i \cdot \frac{e^{-b}/b - e^{-a}/a}{2i(a^2 - b^2)}  \right)
	= \frac{\pi}{a^2 - b^2} \left( \frac{e^{-b}}{b} - \frac{e^{-a}}{a} \right)
\]

\textbf{2}
The function 
\[
	f(z) = \frac{1}{z^2 + 1}
\]
has a pole at $z = i$. The residue of $f(z) e^{iaz}$ here is
\[
	\Res_{z=i} \frac{e^{iaz}}{z^2 + 1}
	= \frac{e^{-a}}{2i}
\]
so
\[
	\int_0^\infty \frac{\cos ax}{x^2 + 1} dx
	= \Re\left( \pi i \cdot \frac{e^{-a}}{2i} \right)
	= \frac{\pi}{2} e^{-a}
\]

\textbf{3}
The function
\[
	f(z) = \frac{1}{(z^2 + b^2)^2} 
\]
has a singular point at $z = ib$. The residue of $f(z) e^{iaz}$ here is
\begin{align*}
	\Res_{z = ib} \frac{e^{iaz}}{(z^2 + b^2)^2}
	&= \frac{d}{dz} \frac{e^{iaz}}{(z + ib)^2} \Big|_{z = ib} \\
	&= \frac{e^{iaz} [ia (z + ib) - 2]}{(z + ib)^3} \Big|_{z = ib} \\
	&= \frac{-2e^{-ab} (1 + ab)}{-8ib^3} \\
	&= \frac{(1 + ab)e^{-ab}}{4ib^3}
\end{align*}
so
\[
	\int_0^\infty f(x) \cos ax~dx
	= \Re \left( \pi i \cdot \frac{(1 + ab)e^{-ab}}{4ib^3} \right)
	= \frac{\pi}{4b^3} (1 + ab)e^{-ab}
\]

\textbf{4}
The function
\[
	f(z) = \frac{z}{z^2 + 3}
\]
tends to zero for large $|z|$. It has a singular point at $z = i\sqrt{3}$, where the residue of $f(z) e^{2iz}$ is
\[
	\Res_{z = i\sqrt{3}} \frac{z e^{2iz}}{z^2 + 3}
	= \frac{z e^{2iz}}{z + i\sqrt{3}} \Big|_{z = i\sqrt{3}}
	= \frac{i\sqrt{3} e^{-2\sqrt{3}}}{2i\sqrt{3}}
	= \frac{e^{-2\sqrt{3}}}{2}
\]
so the integral evaluates to
\[
	\int_0^\infty \frac{x \sin 2x}{x^2 + 3}
	= \Im \left( \pi i \cdot \frac{e^{-2\sqrt{3}}}{2} \right)
	= \frac{\pi}{2} e^{-2\sqrt{3}}
\]





\end{document}